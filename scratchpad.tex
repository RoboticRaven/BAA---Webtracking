% 2-Kolumnen Env
\begin{multicols*}{2}
\end{multicols*}
\newpage

% Verweis auf Grafik in Anhang
(siehe Anhang Abb. \ref{appendix:IUperInhabitant})

% align right
\begin{flushright} \vspace{-7mm} \parencite[]{Metcalfe1995} \end{flushright}

%% TAKE CARE OF THIS --> rename "anote" field to "note"
% mendeley date string for URL 
[abgerufen am 21. Okt. 2021]

% compact list
\begin{itemize}
    \setlength\itemsep{0.05em}
    \item 
\end{itemize}

% in-text cite
\textcite{}

% ref table / figure
\ref{table:research_literature} (S. \pageref{table:research_literature}) 

% TEXT PASSAGES REMOVED  BUT KEPT FOR FUTURE EVENTUAL USE

%\begin{itemize}
%    \setlength\itemsep{0.05em}
%   \item Die Richtlinie 95/46/EG zum Schutz natürlicher Personen bei der Verarbeitung personenbezogener Daten und zum freien Datenverkehr, auch bekannt als 
%  Datenschutzrichtlinie. Diese wurde 2018 durch die Datenschutz-Grundverordnung abgelöst.
%    \item Die Richtlinie 2002/58/EG, auch bekannt als Datenschutzrichtlinie für elektronische Kommunikation oder ePrivacy-Richtlinie
%\end{itemize}
%
% 
%Bereits vor dem Inkrafttreten des Vertrags von Maastricht, welcher das Gründungsdokument der \acrshort{eu} darstellt, ein Vorschlag für eine Datenschutzrichtlinie zum Schutz natürlicher Personen diskutiert. Im Jahr 1995 trat dann die darauf basierende Direktive 95/46/EG in Kraft, welche im Englischen als \emph{Data Protection Directive} \acrshort{dpd} und im Deutschen auch als "Richtlinie 95/46/EG zum Schutz natürlicher Personen bei der Verarbeitung personenbezogener Daten und zum freien Datenverkehr" bekannt ist. Diese erlaubte den Mitgliedsstaaten die enthaltenen Vorgaben mit Implementationsspielraum umzusetzen. \par
%Im Jahr 2018 wurde dann die \acrshort{dpd} durch eine Regulation abgelöst, die sogennante Datenschutz-Grundverordnung (\acrshort{dsgvo}) oder engl. \emph{General Data Protection Regulation} (\acrshort{gdpr}). Als Regulation ist die \acrshort{dsgvo} legal bindend, tritt in allen Mitgliedstaaten gleichermassen in Kraft und führt so zu einer zunehmenden Vereinheitlichung der Gesetzgebung im Bereich des Datenschutzes innerhalb der \acrshort{eu}. \par
%Die DSGVO hat tiefgreifende Auswirkungen darauf, wie Daten in gesetzeskonformer Weise gesammelt, verarbeitet und weitergegeben werden können und legt die Höhe der Strafzahlungen bei Verletzung dieser Prinzipien fest. \par
%Vor allem im Bereich der Werbewirtschaft hat dies auch für Nutzerinnen und Nutzer des Internets im Raum der \acrshort{eu} zu spürbaren Veränderungen geführt: insbe%sondere \emph{Cookie Banners} sind zu einem omnipräsenten Phänomen geworden; man trifft sie beinahe auf jeder Webpräsenz an, in unterschiedlichen Ausführungen im Hinblick auf Aussehen, Informationsgehalt und Vorselektion von Steuerungsmöglichkeiten. \par
%Die Auswirkungen der \acrshort{dsgvo} erstrecken sich jedoch weit über \emph{\gls{cookiebanner}s} hinaus und bilden mitunter Gegenstand der vorliegenden Arbeit.
