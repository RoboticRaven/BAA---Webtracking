% ---------------------------------------- GLOSSARY FILE ----------------------------------------
% -----------------------------------------------------------------------------------------------
%
% ------ ACRONYM ENTRY EXAMPLE:
%\newacronym{gcd}{GCD}{Greatest Common Divisor}
%
%
% ------ GLOSSARY ENTRY EXAMPLE:
%\newglossaryentry{foobar}{
%  name={Foobar},
%  description={A strange animal, not to be confused with \gls{foobar}}
%}
% -----------------------------------------------------------------------------------------------

% -----------------------------------------------------------------------------------------------
% ACRONYMS
% -----------------------------------------------------------------------------------------------
% a
\newacronym{acm}{ACM}{eng. Association for Computing Machinery}
\newacronym{api}{API}{engl. Application Programming Interface, dt. Programmierschnittstelle}
% b
% c
\newacronym{cio}{CIO}{engl. Chief Information Office, dt. Leiter Informationstechnik}
\newacronym{ciso}{CISO}{engl. Chief Information Security Officer, dt. Informationssicherheitsbeauftragter (ISB)}
\newacronym{cnil}{CNIL}{fr. Commission nationale de l'informatique et des libertés , dt. französische Datenschutzbehörde}
% d
\newacronym{dpa}{DPA}{engl. Data Protection Authority, dt. nationaler Datenschutzbeauftragter}
\newacronym{dpd}{DPD}{engl. Data Protection Directive, dt. Datenschutzrichtlinie 95/46/EG}
\newacronym{dns}{DNS}{engl. Domain Name System}
\newacronym{dom}{DOM}{engl. Document Object Model}
\newacronym{dsar}{DSAR}{engl. Data Subject Access Request, dt. Antrag auf die Erteilung einer Auskunft über personenbezogene Daten}
\newacronym{dsgvo}{DSGVO}{dt. Datenschutz-Grundverordnung (engl. GDPR)}
% e
\newacronym{eu}{EU}{dt. Europäische Union, auch engl. European Union}
% f
\newacronym{fr}{FR}{engl. Functional Requirement, dt. funktionale Anforderung}
% g
\newacronym{gdpr}{GDPR}{engl. General Data Protection Regulation (dt. DSGVO)}
\newacronym{gps}{GPS}{engl. \gls{gls-gps} System, dt. Positionsbestimmungssystem}
\newacronym{grch}{GRCh}{dt. Charta der Grundrechte der Europäischen Union, auch EU-Grundrechtecharta}
% h
\newacronym{html}{HTML}{engl. HyperText Markup Language}
% i
\newacronym{isp}{ISP}{engl. Internet Service Provider, dt. Internetanbieter}
\newacronym{ics}{ICS}{dt. Informations- und Cybersicherheit; Abkürzung ist mehrdeutig, wird im Rahmen der Arbeit aber konsequent wie beschrieben ausgelegt}
\newacronym{ieee}{IEEE}{engl. Institute of Electrical and Electronics Engineers}
\newacronym{isms}{ISMS}{engl. Information Security Management System, dt. Informationssicherheitsmanagementsystem}
\newacronym{itu}{ITU}{engl. International Telecommunication Union, dt. internationale Fernmeldeunion}
% j
% k
% l
% m
% n
\newacronym{nfr}{NFR}{engl. Non-Functional Requirement, dt. Nicht-funktionelle Anforderung}
% o
% p
\newacronym{pbd}{PbD}{engl. Privacy by Design}
\newacronym{pet}{PET}{engl. Privacy Enhancing Technologies, dt. Technologien zur Verbesserung des Schutzes der Privatsphäre}
% q
% r
% s
% t
\newacronym{tls}{TLS}{engl. Transport Layer Security}
% u
\newacronym{ui}{UI}{engl. User Interface, dt. Benutzerschnittstelle}
\newacronym{uid}{UID}{engl. User Identifier, dt. eindeutige Benutzerkennung}
\newacronym{url}{URL}{engl. Uniform Resource Locator}
% v
\newacronym{vpn}{VPN}{engl. Virtual Private Network, dt. virtuelles privates Netzwerk}
% w
% x
% y
% z


% -----------------------------------------------------------------------------------------------
% GLS ENTRIES
% -----------------------------------------------------------------------------------------------
% a
\newglossaryentry{gls-api}{
  name={Application Programming Interface},
  description={ist ein Programmteil, der von einem Softwaresystem anderen Programmen zur Anbindung an das System zur Verfügung gestellt wird}
}
\newglossaryentry{gls-acm}{
  name={Association for Computing Machinery},
  description={bezeichnet die erste wissenschaftliche Gesellschaft für Informatik, welche 1947 gegründet wurde. Ziel der ACM ist es, die \enquote{Kunst}, Wissenschaft und Anwendung der Informationstechnologie zu fördern}
}
% b
\newglossaryentry{blacklist}{
  name={Blacklist},
  description={(bzw. \textit{disallowlist, blocklist, denylist}) bezeichnet in der Informatik eine Form der Zugriffskontrolle welche alle Elemente (bspw. E-Mail, URL, Telefonnummer, Namenskombinationen, etc.) ausser spezifizierte Einträge in der \textit{blacklist} zulässt}
}
\newglossaryentry{blockchain}{
  name={Blockchain},
  description={bezeichnet eine kontinuierlich erweiterbare Liste von, durch kryptografische Verfahren zusammengehängten, Datensätzen in einzelnen Blöcken}
}
\newglossaryentry{benchmark}{
  name={Benchmark},
  description={bezeichnet genormte Mess- und Bewertungsverfahren, mit deren Hilfe man die Leistung von EDV-Systemen oder Systemklassen ermitteln und diese nach bestimmten Kriterien vergleichen kann}
}
\newglossaryentry{browserplugins}{
  name={Browser Plugins},
  description={bezeichnen Softwarekomponenten, welche spezifische Funktionalitäten zu einem existierenden Programm hinzufügen. Browser haben in der Vergangenheit auf solche ausführbare Dateien zugegriffen, mittlerweile werden sie grösstenteils abgelehnt. Beispiele wären \textit{Adobe Flash Player, Java virtual machine, QuickTime, Silverlight} und den \textit{Unity Web Player}. Browser Plugins sind zu unterscheiden von Browser Extensions, welche erstere abgelöst haben}
}
% c
\newglossaryentry{cookiebanner}{
  name={Cookie Banner},
  description={ist eine Schnittstelle zur Einholung der Zustimmung von Nutzerinnen und Nutzer zur Nutzung bestimmter Cookies/Tracker}
}
% d
\newglossaryentry{darkpattern}{
  name={Dark Patterns},
  description={bezeichnet ein Benutzerschnittstellen-Design, das darauf ausgelegt ist, den Benutzer zu Handlungen zu verleiten, die dessen Interessen entgegenlaufen.}
}
\newglossaryentry{domain}{
  name={Domain},
  description={(dt. Bereich, Domäne) ist ein zusammenhängender Teilbereich des hierarchischen Domain Name System (DNS)}
}
\newglossaryentry{gls-dns}{
  name={Domain Name System},
  description={ist einer der wichtigsten Dienste in vielen IP-basierten Netzwerken. Seine Hauptaufgabe ist die Beantwortung von Anfragen zur Namensauflösung}
}
% e
\newglossaryentry{eprivacy}{
  name={ePrivacy-Richtlinie},
  description={auch als \emph{cookie law} oder offiziell als "Datenschutzrichtlinie für elektronische Kommunikation" bekannt, regelt seit 2002 verbindliche Mindestvorgaben für den Datenschutz in der Telekommunikation. Seit ihrer Novelle 2009 schränkt die Richtlinie die Verwendung von „Informationen, die im Endgerät eines Teilnehmers gespeichert werden“ (Cookies) ein}
}
\newglossaryentry{ethernet}{
  name={Ethernet},
  description={ist eine Technik, die Software (Protokolle usw.) und Hardware (Kabel, Verteiler, Netzwerkkarten usw.) für kabelgebundene Datennetze spezifiziert, welche ursprünglich für lokale Datennetze (LANs) gedacht war und daher auch als LAN-Technik bezeichnet wird}
}
% f
\newglossaryentry{falsen}{
  name={False Negative},
  description={ist das fehlerhafte Detektieren der Abwesenheit eines binären Zustands (bswp. ein Test zeigt keine virale Inefktion an, aber es liegt eine vor)}
}
\newglossaryentry{falsep}{
  name={False Positive},
  description={ist das fehlerhafte Detektieren der Anwesenheit eines binären Zustands (bspw. ein Test zeigt eine virale Infektions an, aber es liegt keine vor)}
}

% g
\newglossaryentry{gls-gps}{
  name={Global Positioning System},
  description={offiziell auch bekannt als NAVSTAR GPS, ist ein globales Navigationssatellitensystem zur Positionsbestimmung}
}
% h
% i
\newglossaryentry{gls-ieee}{
  name={Institute of Electrical and Electronics Engineers},
  description={ist ein weltweiter Berufsverband von Ingenieuren, Technikern, (Natur-)Wissenschaftlern und angrenzender Berufe hauptsächlich aus den Bereichen Elektrotechnik und Informationstechnik. Es ist Veranstalter von wissenschaftlichen Fachtagungen, Herausgeber diverser akademischer Fachzeitschriften und bildet Gremien für die Standardisierung von Techniken, Hardware und Software}
}
\newglossaryentry{iso27}{
  name={ISO/IEC 27001:2013},
  description={spezifiziert die Anforderungen für die Einrichtung, Implementation, Wartung und kontinuierliche Verbesserung eines Informationssicherheitsmanagementsystems (\acrshort{isms}) innerhalb des Kontexts der Unternehmung}
}
\newglossaryentry{gls-isms}{
  name={Information Security Management System},
  description={(\acrshort{isms}) bezeichnet die Aufstellung von Verfahren und Regeln innerhalb einer Organisation, die dazu dienen, die Informationssicherheit dauerhaft zu definieren, zu steuern, zu kontrollieren, aufrechtzuerhalten und fortlaufend zu verbessern}
}
% j
% k
% l
% m
% n
% o
\newglossaryentry{onionrouting}{
  name={Onion Routing},
  description={bezeichnet eine Technik zum Erreichen von Anonymität im Internet. Hierbei werden die Webinhalte über ständig wechselnde Routen mehrerer Mixe geleitet, welche in diesem Zusammenhang auch Knoten genannt werden}
}
% p
% q
% r
% s
\newglossaryentry{scope}{
  name={Scope},
  description={(dt. Sichtbarkeitsbereich) bezeichnet den Programmabschnitt, in dem eine Variable nutzbar und sichtbar ist. Im Kontext von Web Tracking wird der Begriff hier verwendet, um den Horizont des Beobachtungszeitraums einer Methode abzustecken}
}
\newglossaryentry{staticanalysis}{
  name={Static analysis},
  description={(bzw. \emph{Static Programm Analysis} (dt. Statische Analyse oder Statische Code-Analyse) bezeichnet die Analyse von Computerprogrammen, welche durchgeführt wird ohne das Programm auszuführen}
}
\newglossaryentry{status quo}{
  name={Status quo},
  description={(lateinisch für „bestehender (aktueller) Zustand“, eigentlich „Zustand, in dem …“ oder „Zustand, durch den …“) bezeichnet den gegenwärtigen Zustand einer Sache, der in der Regel zwar problembehaftet ist, bei dem aber die bekannten Möglichkeiten zur Abhilfe ebenfalls problembehaftet sind}
}
\newglossaryentry{string}{
  name={String},
  description={auch Zeichenkette, Zeichenfolge oder Zeichenreihe genannt, bezeichnet in der Informatik eine endliche Folge von Zeichen (z. B. Buchstaben, Ziffern, Sonderzeichen und Steuerzeichen) aus einem definierten Zeichensatz}
}
% t
% u
\newglossaryentry{gls-uid}{
  name={User Identifier},
  description={(UID) bezeichnet eine Kennung deren Zweck es ist Nutzer voneinander unterscheidbar zu machen}
}
\newglossaryentry{gls-url}{
  name={Uniform Resource Locator},
  description={bezeichnet eine Referenz zu einer Web-Ressource, welche Ort (bspw. \texttt{www.google.ch}) und den Mechanismus für den Zugriff beschreibt (bspw. https)}
}
% v
\newglossaryentry{gls-vpn}{
  name={Virtual Private Network},
  description={bezeichnet die, über ein öffentliches Netz hinweg stattfindende, Erweiterung eines privaten Netzwerks. Dies erlaubt es den Geräten in einer Art und Weise zu kommunizieren, als ob sie sich gemeinsam in einem lokalen Netz befinden würden}
}
% w
\newglossaryentry{webcrawler}{
  name={Web Crawler},
  description={(auch Spider, Searchbot oder Robot) bezeichnet ein Computerprogramm, welches automatisch das World Wide Web durchsucht und Webseiten analysiert}
}
\newglossaryentry{webtracking}{
    name={Webtracking},
    description={beschreibt die Vorgehensweise, durch welche Webseitenbetreiber und/oder Drittanbieter, Informationen über Besucher und deren Aktivitäten auf dem Internet sammeln, speichern und teilen}
}
% x
% y
% z






